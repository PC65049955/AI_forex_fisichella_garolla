\section{Related Works}

Numerous hybrid techniques have been examined in preceding years. Based totally at the papers we reviewed, according to the principle set of rules the studies prioritised, the papers can be cut up into going in conjunction with the following categories: regression strategies, optimization strategies, neural networks, and others. Those classes had been made in keeping with the popularity of the principle method of the forecasting system in the past years.

\textbf{TO DO: }
In conclusion, these techniques are not appropriate for all currency pairs and may provide better effects for only a few randomly selected ones, as we can see inside the proposed papers.

Some pattern-based approaches are contradictory to each other.

\subsection{Regression Methods}

Raimund et al. \cite{Raimundo18} proposed a hybrid model for foreign exchange prediction that makes use of wavelet models along with support vector regression (SVR). Before everything, they used a discrete wavelet transform (DWT) technique to interpret facts from their Forex dataset. Then the data were used as the input of support vector regression (SVR) for predicting the foreign exchange prices. They analysed the overall performance of their system with ARIMA and ARFIMA models. The effects confirmed that their system performs higher than ARIMA and ARFIMA models. 

Taveeapiradeecharoen et al. \citep{Taveeapiradeecharoen19} proposed a version for time series inspection and prediction; this is based on compressed vector autoregression. At the start, they used random compression method to decrease a big wide variety of foreign exchange data into a smaller form. After that, they used the Bayesian model averaging (BMA) approach to establish the load of each random compressed datum to attain the intersecting parameters. Their approach can provide out of sample forecasting till fourteen days previous to the real time. They concluded that their system was not suitable to predict all of the 30 Forex currencies. Their proposed study outperformed the existing benchmark of Bayesian autoregression for specific 6 foreign money pairs. 

A huge range of forecasting models have been proposed via the authors of the paper \cite{Serjam18}, by  applying linear kernel SVR to historical data for EUR/USD, GBP/USD, and USD/JPY currency pairs received from high-frequency trading. Previous successive timeframes are used as features to predict the movement of rates in future/next time frame. Upon building models, they found a easy rule that supplied high-quality results.

After reviewing recent papers, it's evident that support vector regression turned into the most used approach included in our reviewed papers. Compressed vector autoregression, the CRT regression tree, and partial least squares regression had been additionally utilized by researchers. However, there are different algorithms which include lasso regression, logistic regression, and multivariate regression which have been abandoned in later years. 
The reviewed literature shows that the system primarily based on a regression model performed higher than ARIMA and ARFIMA models \cite{Raimundo18}, and the model performance may additionally growth \cite{Achchab17} when a regression model is combined with other techniques. However, when operating with a huge number of foreign money pairs, it is able to become hard with regression techniques, as most of the currency pairs return a higher MSE \citep{Taveeapiradeecharoen19}.

\subsection{Optimization Techniques}
Chandrinos et al. \cite{Chandrinos18} proposed a technical system for Forex that was stimulated by using the Donchian channel method. the primary reason in their method become to create profitable portfolios for Forex buying and selling strategy. They first constructed the modified Renko bars (MRBs) via combining their trading guidelines. Their changed MRBs proved to be more correctly responsive than the normal candlesticks used in Forex. They created an optimization level used by eight currency pairs.  To acquire their optimization stage, they used three search-derivative-free global optimization strategies. These algorithms were the swarm optimization algorithm, also referred to as dividing a hyperrectangle (DIRECT), along side multilevel coordinate search (MCS), and pity beetle (PBA). They examined their optimization method and primarily based on the total return they built two kinds of portfolios: an equally weighted portfolio and a Kelly criterion-based portfolio. They evaluated the performance in their approach primarily based at the geometric return, arithmetic mean, and Sharpe ratio. They found out that the proposed version isn't always suitable for three currency pairs, whilst for the others they attain from 29\% until over 200\% general return.

Pradeepkumar et al. \cite{Pradeepkumar17} advised a model for foreign exchange prediction that became primarily based on a quantile regression neural network (QRNN) and particle swarm optimization. They used PSO to train the QRNN and named the version PSO-QRNN. They used 8 pairs currencies. They used seven unique algorithms for the overall performance evaluation of their model: group method of data handling (GMDH), multilayer perceptron (MLP), random forest (RF), a quantile regression neural network (QRNN), generalized autoregressive conditional heteroskedasticity (GARCH), quantile regression random forest (QRRF), and a general regression neural network (GRNN). Once they executed the Diebold–Mariano (DM) evaluation check on all of the test results, they found that their proposed PSO-QRNN version completed higher than all models on  datasets. For the rest of the datasets, QRRF and QRNN carried out better than other approaches.

Das et a. \cite{Das19} proposed a hybrid approach that turned into build the use of extreme learning machine's on-line sequential version and krill herd (KH). The krill herd (KH) was devoted to features reduction. They compared their proposed system with a recurrent backpropagation neural network (RBPNN) and extreme learning machine (ELM). They considered 3 elements: (i) without features reduction (ii) with statistical features reduction, and (iii) with optimized features reduction strategies. For optimized features reduction strategies, they used bacteria foraging optimization (BFO), krill herd, and particle swarm optimization techniques. They used four foreign currency pairs. For RMSE their approach performed first-class. However, in MAE overall performance, their proposed model didn't provide the satisfactory effects.

For foreign exchange buying and selling approach optimization, a genetic set of rules become employed by the authors of the paper  \cite{Galeshchuk17} to evolve a various set of profitable buying and selling rules based totally on weighted moving average approach. They used a time series with 4147 observations inside a range of sixteen years from 2000 to 2015 and they used the close prices of four foreign money pairs. Developed approach yields acceptably high returns on out-of-sample data. The rules acquired using their genetic algorithm result in appreciably better returns than the ones produced by exhaustive search.

In conclusion, these techniques are not appropriate for all currency pairs and may provide better effects for only a few randomly selected ones, as we can see inside the proposed papers.

\subsection{Neural Network}
Ni et al. \cite{Ni19} proposed a model that predicts the time series of foreign exchange using the C-RNN approach. C-RNN uses a convolutional neural network and recurrent neural network. They used a statistics-driven method to study the changing characteristics of Forex. They used the past 10 years records until 2018 for nine currency pairs. 2000 datapoints were contained in their dataset. Using a convolutional neural network and long short-term memory, evaluating RMSE, they discovered that their proposed C-RNN version offers much less mistakes than LSTM and CNN.

Chandrinos et al. \cite{Chandrinos18} proposed a model referred to as the artificial intelligence risk management system (AIRMS) this is based on machine studying. They advanced two risk management structures: One with a neural network (AIRMS-ANN) and the other with the decision tree approach (AIRMS-DT). They used five Forex currencies and the technical indicator and historical time-series data as the input to their proposed model. They divided the output signal into two categories: profitable and not profitable. When they categorized the output signal only as profitable, they were given an growth of 50\% profit over the 2-category labelled version. As assessment metrics, they used the F1 measure for both models. Each AIRMS-ANN and AIRMS-DT carried out well on average and outperformed each other in some cases. Whilst evaluating the Kelly criterion portfolios, the decision tree once more beat the neural network with respect to total return.

Dash et al. \cite{Dash18} proposed a model that makes use of a higher order neural network for Forex prediction. They used a shuffled frog leaping approach with the Pi–sigma neural network for predicting dynamic and non-linear Forex prices. Three currencies had been used for imposing their model. The ISFL algorithm was used for estimating the hidden parameters and improving the prediction price. ISFL algorithm is a progressed model of the shuffled frog leaping algorithm (SFLA) wherein convergence the speed of the network is enhanced along side the predictive potential of the network. They compared the performance of their approach with a range of different ones. Their model furnished higher accuracy in conjunction with higher statistical performance.

Recent study conducted by Ahmed et al. \cite{Ahmed20} suggests that a significant enhancement inside the prediction of Forex prices may be executed through incorporating domain information in the system of training machine learning models. The proposed approach integrates the foreign exchange Loss feature (FLF) into a long short-time memory model called FLF-LSTM, that minimizes the difference between the actual and predictive average of foreign exchange candles. The usage of $10,078$ 4-hour candles of EUR/USD pairs highlighted that, compared to the classic LSTM version, the proposed FLF-LSTM model shows a lower overall mean absolute error rate by 10.96\%.

The neural network became pretty popular in recent years. Many algorithms, inclusive of the modular neural network and deep belief model are yet to be explored. The reviewed literature implies that neural network-primarily based models may be equipped with unique types of procedures, which proves the versatility of those models. but, in some systems, a big network length can also have an effect on the result.

\subsection{Rest of the Methods}
In this section, we present the rest of methods proposed in literature. Far from being an exhaustive collection of all approaches, we wanted to analyse the most recent works in this field and give the reader an overviews of the remaining methods found in literature.

The genetic algorithm and SVM hybrid version had been quite extensively used strategies. The best function of the SVM is that it could be used as a classifier \cite{58} and a regressor for forecasting \cite{59}. The literature advised that after a SVM is incorporated with the genetic algorithm the model can yield a greater return of investment \cite{58}. Nevertheless, on occasion selecting the wrong kernel may offer a huge difference within the end result \cite{59}. Moreover, some approaches rely on the choice of learning model, inputs, and selection mechanisms.

Another perspective is shown by researches exploiting the chaos theory. Studies prove that chaos's extensive applicability can be used in a broader way. Lee et al. \cite{Lee19} indicates that chaos theory can successfully be used as both an economic time series predictor and as a trading strategy optimizer. The issue is deciding on the input parameters. The methods are selected based totally on the dynamics underlying the selected data and what type of evaluation is meant for the system. That makes the system tremendously complex and not always accurate.

Pattern-based techniques have seen pretty a recognition. Recurrent reinforcement learning (RL), dynamic model averaging (DMA), dynamic conditional correlation (DCC), and so forth., were explored through the researchers. Pattern-based methods also proved their flexible adaptability. A few systems can alternate the statistics of the predictors by using the object properties \cite{BARTOS201757} that can be carried out to layout a selection of time series structures. However, some pattern-based approaches are contradictory to each other, as some systems carry out nicely and offer excellent outcomes with a specific algorithm \cite{8376549}, at the same time as other systems perform just the opposite \cite{CONTRERAS20181}. Additionally, some pattern-based models are only capable of predicting the changes over a short-time period and do no longer guarantee success for longer period of prediction time \cite{WILINSKI2019163}.

Finally, the rest of the strategies incorporate several kinds of methods which have been applied for the forecasting of the Forex marketplace. It became determined that Bayesian autoregressive trees (BART), random forest (RF), naive Bayes (NB), ARIMA, and many others have been implemented and explored. A number of these algorithms have been carried out individually, whereas a few have been carried out in a hybrid version. Natural language processing become not often explored, however techniques such as NLP based on sentiment analysis that relies upon on news headlines \cite{Seifollahi} can easily be misguided using wrong news. As a result, right security measures need to be carried out on these methods.

