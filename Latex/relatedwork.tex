\section{Related Works}

Numerous hybrid techniques have been examined in preceding years. Based totally at the papers we reviewed, according to the principle set of rules the studies prioritised, the papers can be cut up into going in conjunction with the following categories: regression strategies, SVM, neural networks, optimisation strategies, chaos theory, sample-based strategies, and others. Those classes had been made in keeping with the popularity of the principle method of the forecasting system in the past years.

\subsection{Regression Methods}

Raimund et al. \cite{Raimundo18} proposed a hybrid model for foreign exchange prediction that makes use of wavelet models along with support vector regression (SVR). Before everything, they used a discrete wavelet transform (DWT) technique to interpret facts from their forex dataset. Then the data were used as the input of support vector regression (SVR) for predicting the foreign exchange prices. They analyzed the overall performance of their system with ARIMA and ARFIMA models. The effects confirmed that their system performs higher than ARIMA and ARFIMA models. 

Taveeapiradeecharoen et al. \citep{Taveeapiradeecharoen19} proposed a version for time series inspection and prediction; this is based on compressed vector autoregression. At the start, they used random compression method to decrease a big wide variety of foreign exchange data into a smaller form. After that, they used the Bayesian model averaging (BMA) approach to establish the load of each random compressed datum to attain the intersecting parameters. Their approach can provide out of sample forecasting till fourteen days previous to the real time. They rconcluded that their system was not suitable to predict all of the 30 forex currencies. Their proposed version proved to have appropriate solution for specific 6 foreign money pairs and outperformed the existing benchmark of Bayesian autoregression. 

A huge range of forecasting models have been proposed via the authors of the paper \cite{Serjam18}, by  applying linear kernel SVR to hostorical data for EUR/USD, GBP/USD, and USD/JPY currency pairs received from high-frequency trading. Previous successive timeframes are used as features to rpedict the movement of rates in future/next time frame. Upon building models, they found a easy rule that supplied high-quality results.

After reviewing recent papers, it's evident that support vector regression turned into the most used approach included in our reviewed papers. Compressed vector autoregression, the CRT regression tree, and partial least squares regression had been additionally utilized by researchers. However, there are different algorithms which include lasso regression, logistic regression, and multivariate regression which have been abandoned in later years. 
The reviewed literature shows that the system primarily based on a regression model performed higher than ARIMA and ARFIMA models \cite{Raimundo18}, and the model performance may additionally growth \cite{Achchab17} when a regression model is combined with other techniques. However, when operating with a huge number of foreign money pairs, it is able to become hard with regression techniques, as most of the currency pairs return a higher MSE \citep{Taveeapiradeecharoen19}.

\subsection{Optimization Techniques}

\subsection{Neural Network}

\subsection{Rest of the Methods}
