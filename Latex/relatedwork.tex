\section{Related Works}

Numerous hybrid techniques have been examined in preceding years. Based totally at the papers we reviewed, according to the principle set of rules the studies prioritised, the papers can be cut up into going in conjunction with the following categories: regression strategies, optimization strategies, neural networks, and others. Those classes had been made in keeping with the popularity of the principle method of the forecasting system in the past years.

\subsection{Regression Methods}

Raimund et al. \cite{Raimundo18} proposed a hybrid model for foreign exchange prediction that makes use of wavelet models along with support vector regression (SVR). Before everything, they used a discrete wavelet transform (DWT) technique to interpret facts from their forex dataset. Then the data were used as the input of support vector regression (SVR) for predicting the foreign exchange prices. They analyzed the overall performance of their system with ARIMA and ARFIMA models. The effects confirmed that their system performs higher than ARIMA and ARFIMA models. 

Taveeapiradeecharoen et al. \citep{Taveeapiradeecharoen19} proposed a version for time series inspection and prediction; this is based on compressed vector autoregression. At the start, they used random compression method to decrease a big wide variety of foreign exchange data into a smaller form. After that, they used the Bayesian model averaging (BMA) approach to establish the load of each random compressed datum to attain the intersecting parameters. Their approach can provide out of sample forecasting till fourteen days previous to the real time. They rconcluded that their system was not suitable to predict all of the 30 forex currencies. Their proposed study outperformed the existing benchmark of Bayesian autoregression for specific 6 foreign money pairs. 

A huge range of forecasting models have been proposed via the authors of the paper \cite{Serjam18}, by  applying linear kernel SVR to hostorical data for EUR/USD, GBP/USD, and USD/JPY currency pairs received from high-frequency trading. Previous successive timeframes are used as features to rpedict the movement of rates in future/next time frame. Upon building models, they found a easy rule that supplied high-quality results.

After reviewing recent papers, it's evident that support vector regression turned into the most used approach included in our reviewed papers. Compressed vector autoregression, the CRT regression tree, and partial least squares regression had been additionally utilized by researchers. However, there are different algorithms which include lasso regression, logistic regression, and multivariate regression which have been abandoned in later years. 
The reviewed literature shows that the system primarily based on a regression model performed higher than ARIMA and ARFIMA models \cite{Raimundo18}, and the model performance may additionally growth \cite{Achchab17} when a regression model is combined with other techniques. However, when operating with a huge number of foreign money pairs, it is able to become hard with regression techniques, as most of the currency pairs return a higher MSE \citep{Taveeapiradeecharoen19}.

\subsection{Optimization Techniques}
Chandrinos et al. \cite{Chandrinos18} proposed a technical system for FOREX that was stimulated by using the Donchian channel method. the primary reason in their method become to create profitable portfolios for FOREX buying and selling strategy. They first constructed the modified Renko bars (MRBs) via combining their trading guidelines. Their changed MRBs proved to be more correctly responsive than the normal candlesticks used in FOREX. They created an optimization level used by eight currency pairs.  To acquire their optimization stage, they used three search-derivative-free global optimization strategies. These algorithms were the swarm optimization algorithm, also referred to as dividing a hyperrectangle (DIRECT), along side multilevel coordinate search (MCS), and pity beetle (PBA). They examined their optimization method and primarily based on the total return they built two kinds of portfolios: an equally weighted portfolio and a Kelly criterion-based portfolio. They evaluated the performance in their approach primarily based at the geometric return, arithmetic mean, and Sharpe ratio. They found out that the proposed version isn't always suitable for three currency pairs, whilst for the others they attain from 29\% until over 200\% general return.

Pradeepkumar et al. \cite{Pradeepkumar17} advised a model for foreign exchange prediction that became primarily based on a quantile regression neural network (QRNN) and particle swarm optimization. They used PSO to train the QRNN and named the version PSO-QRNN. They used 8 pairs currencies. They used seven unique algorithms for the overall performance evaluation of their model: group method of data handling (GMDH), multilayer perceptron (MLP), random forest (RF), a quantile regression neural network (QRNN), generalized autoregressive conditional heteroskedasticity (GARCH), quantile regression random forest (QRRF), and a general regression neural network (GRNN). Once they executed the Diebold–Mariano (DM) evaluation check on all of the test results, they found that their proposed PSO-QRNN version completed higher than all models on  datasets. For the rest of the datasets, QRRF and QRNN carried out better than other approaches.

Das et a. \cite{Das19} proposed a hybrid approach that turned into build the use of extreme learning machine's on-line sequential version and krill herd (KH). The krill herd (KH) was devoted to features reduction. They compared their proposed system with a recurrent backpropagation neural network (RBPNN) and extreme learning machine (ELM). They considered 3 elements: (i) without features reduction (ii) with statistical features reduction, and (iii) with optimized features reduction strategies. For optimized features reduction strategies, they used bacteria foraging optimization (BFO), krill herd, and particle swarm optimization techniques. They used four foreign currency pairs. For RMSE their approach performed first-class. However, in MAE overall performance, their proposed model didn't provide the satisfactory effects.

For foreign exchange buying and selling approach optimization, a genetic set of rules become employed by the authors of the paper  \cite{Galeshchuk17} to evolve a various set of profitable buying and selling rules based totally on weighted moving average approach. They used a time series with 4147 observations inside a range of sixteen years from 2000 to 2015 and they used the close prices of four foreign money pairs. Developed approach yields acceptably high returns on out-of-sample data. The rules acquired using their genetic algorithm result in appreciably better returns than the ones produced by exhaustive search.

In conclusion, these techniques are not appropriate for all currency pairs and may provide better effects for only a few randomly selected ones, as we can see inside the proposed papers.

\subsection{Neural Network}

\subsection{Rest of the Methods}
